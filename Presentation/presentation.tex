% !TEX encoding = UTF-8 Unicode
\RequirePackage[l2tabu, orthodox]{nag}
\RequirePackage{silence}
\documentclass[english, french]{beamer}
\input{preamble/packages}
\input{preamble/math_basics}
\input{preamble/math_mine}
\input{preamble/redac}
\input{preamble/draw}
\input{preamble/acronyms}
\input{preamble/uml}
\input{preamble/refAPIcmds}

\setbeamertemplate{headline}[singleline]
\setbeamertemplate{footline}[onlypage]

\title{Présentation J-Voting}
\subtitle{}
\subject{Java}
\keywords{}
\author{Adèle Thorner, Laetitia Courgey, \\Vincent Navales, Quentin Sauvage}
\institute{L3 MIAGE Apprentissage, Université Paris-Dauphine}
\date{Version du \today}

\begin{document}
\bibliographystyle{apalike}

\begin{frame}[plain]
	\tikz[remember picture,overlay]{
		\path (current page.south) ++ (0, 1mm) node[anchor=south, inner sep=0] {
			\includegraphics[height=9mm]{Dauphine.jpg}
		};
	}
   \titlepage
\end{frame}
\addtocounter{framenumber}{-1}

\section{Introduction}
\begin{frame}
	\frametitle{Plan de la présentation}
	\tableofcontents
\end{frame}

\begin{frame}
    \frametitle{Introduction}
    \begin{columns}
        \begin{column}{0.5\textwidth}
            {\Large Théorie du choix social}
            \begin{itemize}
                \item Problèmes de décisions collectives
                \item Agrégation de préférences\\ $\rightarrow$ déduction d'une préférence sociale
            \end{itemize}
        \end{column}
        
        \pause
        \begin{column}{0.5\textwidth}
            {\Large L'application J-Voting}
            \begin{itemize}
                \item Visualisation de profil d'élection
                \item Modification de profil
            \end{itemize}
        \end{column}
    \end{columns}
\end{frame}

\section{Fonctionnement de l'application}
	\subsection{Représentation d'une élection avec un profil}
	    \begin{frame}
        	\frametitle{Représentation d'une élection avec un profil}
        	\begin{itemize}
        	    \item Structure d'un profil
        	    \begin{itemize}
        	        \item[o] Voteurs
        	        \item[o] Préférences constituées d’alternatives ordonnées
        	    \end{itemize}
        	    \item Différents types de profils
        	    \begin{itemize}
        	        \item[o] Complet / Incomplet
        	        \item[o] Strict / non Strict
        	    \end{itemize}
        	\end{itemize}
        \end{frame}
    \subsection{Construction et modification de profils}
	    \begin{frame}
        	\frametitle{Construction et modification de profils}
        	    \begin{itemize}
        	        \item 2 classes pour construire les profils :
        	        \begin{itemize}
        	            \item[o] ProfileBuilder
        	            \item[o] StrictProfileBuilder
        	        \end{itemize}
        	        \item Lecture de fichiers .soc, .soi (.toc, .toi non implémenté) depuis plusieurs sources :
        	       \begin{itemize}
        	           \item[o] En local
        	           \item[o] Depuis une URL
        	           \item[o] Depuis une table affichée avec une GUI
        	       \end{itemize}
        	    \end{itemize}
        \end{frame}
    \subsection{Analyse de profils}
	    \begin{frame}
        	\frametitle{Analyse de profils}
            	\begin{itemize}
            	    \item Une interface SocialWelfareFunction
            	    \begin{itemize}
            	        \item[o] Borda
            	        \item[o] Dictator
            	        \item[o] FrenchElection
            	    \end{itemize}
            	\end{itemize}
        \end{frame}
	\subsection{Interactions entre les classes}
        \begin{frame}
        	\frametitle{Interactions entre les classes}
        	\begin{center}
        	    Diagramme de classes
        	    \includegraphics[width = 0.6\textwidth]{"graphics/19 - Diagramme classes".PNG}
        	\end{center}
        \end{frame}

\section{Démonstration}
	\subsection{Lire un profil}
	    \begin{frame}
        	\frametitle{Lire un profil}
        	\begin{center}
        	    Affichage en colonnes d'un profil au format SOC
        	    \includegraphics[width = \textwidth]{"graphics/04 - SOC Columns Display".PNG}
        	\end{center}
        \end{frame}
        
	\subsection{Ecrire un profil}
	    \begin{frame}
        	\frametitle{Ecrire un profil}
        	\begin{center}
        	    Modification d'un profil au format SOC\\
        	    \includegraphics[width = 0.5\textwidth]{"graphics/18 - Add Alternative - Save buttons".PNG}
        	\end{center}
        \end{frame}

\section{Travail de groupe}
    \begin{frame}
        \frametitle{Travail de groupe}     
        \begin{itemize}
            \item {\Huge Binômes tournants}\vspace{1cm}
            \item {\Huge Difficultés}
        \end{itemize}
    \end{frame}

\section{Conclusion}
    \begin{frame}
        \frametitle{Conclusion}        
        \begin{itemize}
            \item Le travail effectué \pause
            \item Le travail qu'on aurait souhaité effectuer \pause
            \item Les compétences acquises
        \end{itemize}
    \end{frame}

\end{document}
